\subsection{Ideas de J. W. Gibbs: Desarrollo en potencias de la densidad}

La expresión presenta, esencialemente, los dos primer términos del desarrollo
de potencias de $1/V$

\begin{equation}
    P = \frac{NT}{V}\left(
        1 + \frac{NB(\beta)}{V} + \frac{N^2 C(\beta)}{V^2} + \ldots
    \right)
\end{equation}

como ya se mensiono, el primer término reprecenta a un gas perfecto,
con el segundo termino a particulas interactuantes par y agregando 
el tercer término para particulas interactuantes cuando son 3.

Para deducir estos coeficientes podemos hacer lo siguiente.

Para N = 1, ya se conoce
\begin{equation}
    E_1(p, 1) = p^2/2m
\end{equation}

Para $N = 2$, se tiene la energía cinética más la energía
de su interacción $U_{12}$

\begin{equation}
    E_2(p, q) = \sum_{i=1}^{2} \frac{p_i^2}{2m} + {U_{12}}
\end{equation}

De igual forma para $N = 3$ con energía de interacción $U_{123}$

\begin{equation}
    E_3(p, q) = \sum_{i=1}^{3} \frac{p_i^2}{2m} + {U_{123}}
\end{equation}


\parencite[p.~124]{landau} Define la formula para el potencial $\Omega$
como (Se a moduficado la temperatura por $\beta = kT$)
\begin{equation}
    \Omega = -T \ln \left[
        \sum_{N}e^{-T\mu N}\int e^{-T E_N(p, 1) d\Gamma_N}
    \right]
\end{equation}

remplazando esta la penultima ecuación en esta, usando la notación

\begin{equation}
    \xi = \frac{e^{\mu/T}}{{(2\pi\hbar)}^2}
    \int e^{ p^2/2m T} d^3p 
    = {\left(\frac{mT}{2\pi\hbar^2}\right)}^{3/2}e^{\mu/T}
\end{equation}

Así podemos obtener

\begin{equation}
    \Omega = -T \ln \left[
        1 + \xi V + \frac{\xi^2}{2!}
        \int\int e^{-\beta U_{12}}dV_1 dV_2
        + \frac{\xi^3}{3!} \int\int\int e^{-\beta U_{123}}dV_1 dV_2 dV_3
        + \cdots
    \right]
\end{equation}

Si medimos con respecto al primer átomo, se reduce la multiplicidad de
la integrarl así se tiene

\begin{equation}
    \Omega = -PV = -T \ln \left[
        1 + \xi V + \frac{\xi^2}{2!}
        \int e^{-\beta U_{12}}dV_2
        + \frac{\xi^3}{3!} \int\int e^{-\beta U_{123}}dV_2 dV_3
        + \cdots
    \right]
\end{equation}


Si desarrollamos esta serie en potencias de $\xi$ se tiene:

\begin{equation}
    P = T \sum_{n = 1}^{\infty}\frac{J_n}{n!}\xi^n
\end{equation}

Donde
\begin{gather}
    J_1 = 1,\qquad
    j_2 = \int \left(e^-{\beta U_{12} - 1}\right)\\
    j_3 = \int\int(e^{-\beta\mu_{123}} - e^{-\beta U_{12}} - 
        e^{-\beta U_{13}} + e^{-\beta U_{23}}) dV_2 dV_3
\end{gather}

Derivando con respecto a $\mu$, se obtiene el número de partículas del gas,
puesto que

\begin{equation}
    N = - {\left(
        \frac{\partial \Omega}{\partial \mu}
    \right)}_{T, V}
    = V {\left(
        \frac{\partial P}{\partial \mu}
    \right)}_{T, V}
\end{equation}

De la definición de la ecuación 18 se tiene $\partial \xi \partial \mu = \xi/T$

finalmente se obtiene

\begin{equation}
    N = V \sum_{n = 1}^{\infty}\frac{J_n}{(n - 1)!}\xi^n
\end{equation}

