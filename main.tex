\documentclass[12pt, a4paper]{article}
\usepackage{amsmath}
\usepackage{amssymb}
\usepackage[utf8]{inputenc}    % Soporte para caracteres especiales
\usepackage{amsfonts}
\usepackage[right=3cm, left=2.5cm, top=3cm, bottom=3.5cm]{geometry}
% \usepackage{hyperref}          % Enlaces y referencias
\usepackage[style=apa,backend=biber]{biblatex}
\usepackage{csquotes}          % Manejo de citas (opcional, útil para biblatex)
\addbibresource{refs.bib}
\usepackage[spanish]{babel}

\title{Sistemas formados por par partículas interactuantes, Ideas de J. W. Gibbs}

\author{Carrasco Soto Miguel Alberto, \\Torres Tarrillo Rober Esbel}

\begin{document}

\maketitle
\tableofcontents

\newpage
\section{Introducción}
Las interacciones entre las partículas, que pueden ser de diversa naturaleza (como las fuerzas electromagnéticas, gravitacionales o de Van der Waals), introducen correlaciones espaciales y temporales entre ellas, lo que hace que el análisis estadístico de estos sistemas sea mucho más complicado. Este tipo de sistemas es común en líquidos, gases reales, sólidos y plasmas, donde las interacciones juegan un papel crucial en la determinación de las propiedades termodinámicas del sistema.

El estudio de estos sistemas se apoya en conceptos clave de la mecánica estadística, como los ensambles estadísticos, las funciones de partición, las funciones de distribución y las correlaciones de pares. Mediante estas herramientas, es posible conectar las propiedades microscópicas, como las posiciones y momentos de las partículas, con las macroscópicas, tales como la presión, la temperatura y la energía interna.

En este contexto, el trabajo de Josiah Willard Gibbs, al introducir el concepto de ensamble estadístico y la relación entre las mecánicas microscópica y macroscópica, resulta fundamental. Su desarrollo de métodos estadísticos para describir sistemas con interacciones ha permitido avances en la comprensión de fenómenos complejos, desde el comportamiento de gases reales hasta la dinámica de líquidos y sólidos.
\section{Objetivos}
\subsection{General}
    Encontrar las ecuaciones de estado para sistemas formados por partículas
    interactuanes

\subsection{Específicos}
    \begin{itemize}
    \item Encontrar la función de estado para un sistema simplicado con 
    un solo par de partículas interactuantes.
    
    \item A partir de las ideas simplificadas generalizar a interacción
    par entre multiples partículas.

    \end{itemize}
    
\section{Gases Reales}
\subsection{Gases Reales}
La ecuación de estado de un gas perfecto puede aplicarse
frecuentemente y con suficiente precisión a los gases reales. Sin embargo, esta aproximación
puede resultar insuficiente y surge entonces la necesidad de tener en cuenta las desviaciones de
gas real respecto de un gas perfecto debidas a la interacción de las moléculas que lo constituyen.
\parencite[p.~261]{landau}

Tomando las condiciones de~\parencite[p.~261]{landau} 
``\ldots considerando el gas hasta tal punto enrarecido, que sea posible prescindir de las colisiones ternarias, cuaternarias, etc.,
de las moléculas entre sí suponer que su interacción se efectúa solamente mediante colisiones binarias.''

Con esto en mente consideremos primero un gas mono-atómico así la energía total de $N$ átomos
del sistema será

\begin{equation}
    \label{eq:energia-total}
    E(p, q) = \sum_{i=1}^N \frac{p_i^2}{2m} + U
\end{equation}

donde $p$ es el momento lineal de cada partícula y $U = U(q)$ es energía
de interacción dos a dos.

De las deducciones para un gas ideal clásicos deducidas por~\parencite[p.~205, traducción propia]{Swendsen2012} ``
\ldots la integral es conocida la transformada de Laplace. Es similar a a la
transformada de Fourier, pero el factor en el exponente es real, en vez de imaginario''

\begin{equation}
    \label{eq:partition}
    Z(T, V, N) = \int \Omega(E, V, N) e^{-\beta E} dE
\end{equation}

Donde $T$ es la temperatura, $V$ el volumen $N$ el número de partículas del sistema, Energía
del sistema y $\beta = 1/kT$, $k$ es la constante de Boltzmann.

Solo es interés la integral
\begin{equation}
    Z = \int e^{-\beta E} dE
\end{equation}

Por lo que remplazando~\ref{eq:energia-total} en la ecuación anterior y
solo quedados con la integral para $U$
\begin{equation}
    \int \cdots \int e^{-\beta U} dV_1dV_2 \ldots dV_N,
\end{equation}

donde $dV_i = dx_i dy_i dz_i$. Si $U = 0$ entonces esta integral sería
$V^N$ y como la integral del primer término de~\ref{eq:energia-total}
es la energía del gas perfecto al cual llamaremos $F_p$ así
usando las propiedades de logaritmos podemos hacer

\begin{equation}
    F = F_p - T \ln \frac{1}{V^N} \int\ldots\int e^{-\beta U} dV_1dV_2 \ldots dV_N
\end{equation}

Sumando uno y restando uno a la expresión anterior

\begin{equation}
    F = F_p - T \ln  \left[\frac{1}{V^N} \int\ldots\int \left(e^{-\beta U}  - 1\right)dV_1dV_2 \ldots dV_N + 1\right]
\end{equation}

Para calcular los cálculos faltantes vamos a usar la suposiciones de
\parencite[p.~262]{landau} ``Supondremos que el gas no solo está suficientemente
enrarecido, si también que la cantidad del mismo es suficientemente pequeña 
como para que se pueda suponer que en él no chocan a la vez más de un par de átomos.
Con esto se evita el problema de las colisiones de más de dos átomos los cuales solo
harían que el problema se agrande.

Por otra  parte la forma de elegir estos átomos está dado por
\begin{equation}
    \binom{N}{2} = \frac{N!}{(N - 2)! 2!} = \frac{1}{2}N(N - 1)
\end{equation}

Así la integral anterior se puede escribir como

\begin{equation}
    \frac{1}{2}N(N - 1)
    \int\ldots\int \int\ldots\int \left(e^{-\beta U_{12}}  - 1\right)dV_1dV_2 \ldots dV_N
\end{equation}

Donde $U_{12}$ es la aspergía de dos átomos (sin importar cuales sean debido
a su identidad) ademas podemos escribir $N^2$ debido a que $N$ es muy grande,
ademas utilizando el hecho que $\ln(1 + x) \approx x$
así obtenemos

\begin{equation}
    F = F_p - \frac{TN^2}{2V^2}\int\int(e^{-\beta U_{12}} - 1) dV_1 dV_2
\end{equation}

Si en vez de las coordenadas de los dos átomos se introduces las coordenadas
de su centro de masa común y sus condenadas relativas, entonces
$U_{12}$ solo dependerá de estas dos últimas la cual podemos representar por
$dV$ y como podemos integrar con respecto al centro de masas común lo
cual dará un nuevo volumen así obtenemos

\begin{equation}
    F = F_p + \frac{N^2TB(\beta)}{V}
\end{equation}

donde
\begin{equation}
    B(T) = \frac{1}{2}\int\left(1 - e^{-\beta U_{12}}\right)dV
\end{equation}

Como la presión está dada por $P = - \partial F / \partial V$ entonces

\begin{equation}
    P = \frac{N k_b T}{V} \left(1 + \frac{NB(\beta)}{V}\right)
\end{equation}
\subsection{Ideas de J. W. Gibbs: Desarrollo en potencias de la densidad}

La expresión presenta, esencialmente, los dos primer términos del desarrollo
de potencias de $1/V$

\begin{equation}
    P = \frac{NT}{V}\left(
        1 + \frac{NB(\beta)}{V} + \frac{N^2 C(\beta)}{V^2} + \ldots
    \right)
\end{equation}

como ya se mencionó, el primer término representa a un gas perfecto,
con el segundo termino a partículas interactuantes par y agregando 
el tercer término para partículas interactuantes cuando son 3.

Para deducir estos coeficientes podemos hacer lo siguiente.

\parencite[p.~124]{landau} Define la formula para el potencial $\Omega$
como (Se a modificado la temperatura por $\beta = kT$)
\begin{equation}
    \label{eq:potential}
    \Omega = -T \ln \left[
        \sum_{N}\frac{e^{-\beta\mu N}}{N!}\int e^{-\beta E_N(p, q)} d\Gamma_N
    \right]
\end{equation}



Para $N = 0$ Se tiene que $E_0 = 0$

Para N = 1, Solo se tiene la energía cinética
\begin{equation}
    E_1(p, 1) = p^2/2m
\end{equation}

Entonces Se tiene

\begin{align*}
    e^{\beta\mu} \int e^{-\beta E_N(p, q)} d\Gamma_N
     &= \frac{e^{\beta\mu}}{{(2\pi\hbar)}^3}  \int e^{-\beta p^2/2m} dp
     = \frac{e^{\beta\mu}}{{(2\pi\hbar)}^3} \sqrt{\frac{2 \pi m}{\beta}}^3\\
     &= {\left(\frac{m}{2\pi\hbar^2}\right)}^{3/2} e^{\beta\mu}
\end{align*}

Para $N = 2$, se tiene la energía cinética más la energía
de su interacción $U_{12}$

\begin{equation}
    E_2(p, q) = \sum_{i=1}^{2} \frac{p_i^2}{2m} + {U_{12}}
\end{equation}

Por lo que el tercer sumando será

\begin{align}
   e^{2\beta\mu} \int e^{-\beta E_2(p, q)} d\Gamma_N
   &= \frac{e^{\beta\mu}}{{(2\pi\hbar)}^3}^2
    \int e^{-\beta p_1^2/2m} dp_1
    \int e^{-\beta p_2^2/2m} dp_2
   \int \int e^{-\beta U_{12}} dV_1 dV_2\\
    &= {{\left(\frac{m}{2\pi\hbar^2}\right)}^{3/2} e^{\beta\mu}}^2
\int \int e^{-\beta U_{12}} dV_1 dV_2
\end{align}



De igual forma para $N = 3$ con energía de interacción $U_{123}$

\begin{equation}
    E_3(p, q) = \sum_{i=1}^{3} \frac{p_i^2}{2m} + {U_{123}}
\end{equation}


sustituyendo y haciendo las integrales (es de la misma forma)
remplazando esta la penúltima ecuación en esta, usando la notación

\begin{equation}
    \xi = \frac{e^{\mu/T}}{{(2\pi\hbar)}^2}
    \int e^{ p^2/2m T} d^3p 
    = {\left(\frac{mT}{2\pi\hbar^2}\right)}^{3/2}e^{\mu/T}
\end{equation}

Así podemos obtener

\begin{equation}
    \Omega = -T \ln \left[
        1 + \xi V + \frac{\xi^2}{2!}
        \int\int e^{-\beta U_{12}}dV_1 dV_2
        + \frac{\xi^3}{3!} \int\int\int e^{-\beta U_{123}}dV_1 dV_2 dV_3
        + \cdots
    \right]
\end{equation}

Si medimos con respecto al primer átomo, se reduce la multiplicidad de
la integral así se tiene

\begin{equation}
    \Omega = -PV = -T \ln \left[
        1 + \xi V + \frac{\xi^2}{2!}
        \int e^{-\beta U_{12}}dV_2
        + \frac{\xi^3}{3!} \int\int e^{-\beta U_{123}}dV_2 dV_3
        + \cdots
    \right]
\end{equation}


Si desarrollamos esta serie en potencias de $\xi$ se tiene:

\begin{equation}
    P = T \sum_{n = 1}^{\infty}\frac{J_n}{n!}\xi^n
\end{equation}

Donde
\begin{gather}
    J_1 = 1,\qquad
    j_2 = \int \left(e^-{\beta U_{12} - 1}\right)\\
    j_3 = \int\int(e^{-\beta\mu_{123}} - e^{-\beta U_{12}} - 
        e^{-\beta U_{13}} + e^{-\beta U_{23}}) dV_2 dV_3
\end{gather}

Derivando con respecto a $\mu$, se obtiene el número de partículas del gas,
puesto que

\begin{equation}
    N = - {\left(
        \frac{\partial \Omega}{\partial \mu}
    \right)}_{T, V}
    = V {\left(
        \frac{\partial P}{\partial \mu}
    \right)}_{T, V}
\end{equation}

De la definición de la ecuación 18 se tiene $\partial \xi \partial \mu = \xi/T$

finalmente se obtiene

\begin{equation}
    N = V \sum_{n = 1}^{\infty}\frac{J_n}{(n - 1)!}\xi^n
\end{equation}



\section{Conclusiones}

\begin{itemize}
\item El estudio de partículas interactuares se vuelve complejo, por lo que
se debe simplificar a un sistema pequeño y simplificado para luego a partir
de este poder generalizar estas ecuaciones a sistemas un poco más complejos
aprovecho el carácter aditivo de la energía libre.

\item El desarrollo por potencias de densidad fue un método inspirado por
Gibs para escribir cómo una serie las interactuantes.
\end{itemize}


\printbibliography 
\end{document}