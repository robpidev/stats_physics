\section{Introducción}
Las interacciones entre las partículas, que pueden ser de diversa naturaleza (como las fuerzas electromagnéticas, gravitacionales o de Van der Waals), introducen correlaciones espaciales y temporales entre ellas, lo que hace que el análisis estadístico de estos sistemas sea mucho más complicado. Este tipo de sistemas es común en líquidos, gases reales, sólidos y plasmas, donde las interacciones juegan un papel crucial en la determinación de las propiedades termodinámicas del sistema.

El estudio de estos sistemas se apoya en conceptos clave de la mecánica estadística, como los ensambles estadísticos, las funciones de partición, las funciones de distribución y las correlaciones de pares. Mediante estas herramientas, es posible conectar las propiedades microscópicas, como las posiciones y momentos de las partículas, con las macroscópicas, tales como la presión, la temperatura y la energía interna.

En este contexto, el trabajo de Josiah Willard Gibbs, al introducir el concepto de ensamble estadístico y la relación entre las mecánicas microscópica y macroscópica, resulta fundamental. Su desarrollo de métodos estadísticos para describir sistemas con interacciones ha permitido avances en la comprensión de fenómenos complejos, desde el comportamiento de gases reales hasta la dinámica de líquidos y sólidos.