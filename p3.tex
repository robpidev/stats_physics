\subsection{Gases Reales}
La ecuación de estado de un gas perfecto puede aplicarse
frecuentemente y con suficiente precisión a los gases reales. Sin embargo, esta aproximación
puede resultar insuficiente y surge entonces la necesidad de tener en cuenta las desviaciones de
gas real respecto de un gas perfecto debidas a la interacción de las moléculas que lo constituyen.
\parencite[p.~261]{landau}

Tomando las condiciones de~\parencite[p.~261]{landau} 
``\ldots considerando el gas hasta tal punto enrarecido, que sea posible prescindir de las colisiones ternarias, cuaternarias, etc.,
de las moléculas entre sí suponer que su interacción se efectúa solamente mediante colisiones binarias.''

Con esto en mente consideremos primero un gas mono-atómico así la energía total de $N$ átomos
del sistema será

\begin{equation}
    \label{eq:energia-total}
    E(p, q) = \sum_{i=1}^N \frac{p_i^2}{2m} + U
\end{equation}

donde $p$ es el momento lineal de cada partícula y $U = U(q)$ es energía
de interacción dos a dos.

De las deducciones para un gas ideal clásicos deducidas por~\parencite[p.~205, traducción propia]{Swendsen2012} ``
\ldots la integral es conocida la transformada de Laplace. Es similar a a la
transformada de Fourier, pero el factor en el exponente es real, en vez de imaginario''

\begin{equation}
    \label{eq:partition}
    Z(T, V, N) = \int \Omega(E, V, N) e^{-\beta E} dE
\end{equation}

Donde $T$ es la temperatura, $V$ el volumen $N$ el número de partículas del sistema, Energía
del sistema y $\beta = 1/kT$, $k$ es la constante de Boltzmann.

Solo es interés la integral
\begin{equation}
    Z = \int e^{-\beta E} dE
\end{equation}

Por lo que remplazando~\ref{eq:energia-total} en la ecuación anterior y
solo quedados con la integral para $U$
\begin{equation}
    \int \cdots \int e^{-\beta U} dV_1dV_2 \ldots dV_N,
\end{equation}

donde $dV_i = dx_i dy_i dz_i$. Si $U = 0$ entonces esta integral sería
$V^N$ y como la integral del primer término de~\ref{eq:energia-total}
es la energía del gas perfecto al cual llamaremos $F_p$ así
usando las propiedades de logaritmos podemos hacer

\begin{equation}
    F = F_p - T \ln \frac{1}{V^N} \int\ldots\int e^{-\beta U} dV_1dV_2 \ldots dV_N
\end{equation}

Sumando uno y restando uno a la expresión anterior

\begin{equation}
    F = F_p - T \ln  \left[\frac{1}{V^N} \int\ldots\int \left(e^{-\beta U}  - 1\right)dV_1dV_2 \ldots dV_N + 1\right]
\end{equation}

Para calcular los cálculos faltantes vamos a usar la suposiciones de
\parencite[p.~262]{landau} ``Supondremos que el gas no solo está suficientemente
enrarecido, si también que la cantidad del mismo es suficientemente pequeña 
como para que se pueda suponer que en él no chocan a la vez más de un par de átomos.
Con esto se evita el problema de las colisiones de más de dos átomos los cuales solo
harían que el problema se agrande.

Por otra  parte la forma de elegir estos átomos está dado por
\begin{equation}
    \binom{N}{2} = \frac{N!}{(N - 2)! 2!} = \frac{1}{2}N(N - 1)
\end{equation}

Así la integral anterior se puede escribir como

\begin{equation}
    \frac{1}{2}N(N - 1)
    \int\ldots\int \int\ldots\int \left(e^{-\beta U_{12}}  - 1\right)dV_1dV_2 \ldots dV_N
\end{equation}

Donde $U_{12}$ es la aspergía de dos átomos (sin importar cuales sean debido
a su identidad) ademas podemos escribir $N^2$ debido a que $N$ es muy grande,
ademas utilizando el hecho que $\ln(1 + x) \approx x$
así obtenemos

\begin{equation}
    F = F_p - \frac{TN^2}{2V^2}\int\int(e^{-\beta U_{12}} - 1) dV_1 dV_2
\end{equation}

Si en vez de las coordenadas de los dos átomos se introduces las coordenadas
de su centro de masa común y sus condenadas relativas, entonces
$U_{12}$ solo dependerá de estas dos últimas la cual podemos representar por
$dV$ y como podemos integrar con respecto al centro de masas común lo
cual dará un nuevo volumen así obtenemos

\begin{equation}
    F = F_p + \frac{N^2TB(\beta)}{V}
\end{equation}

donde
\begin{equation}
    B(T) = \frac{1}{2}\int\left(1 - e^{-\beta U_{12}}\right)dV
\end{equation}

Como la presión está dada por $P = - \partial F / \partial V$ entonces

\begin{equation}
    P = \frac{N k_b T}{V} \left(1 + \frac{NB(\beta)}{V}\right)
\end{equation}